\documentclass[a4paper]{scrartcl}
\usepackage[ngerman]{babel}
\usepackage[T1]{fontenc}
\usepackage[utf8]{inputenc}
\usepackage{graphicx}
\usepackage{color}
\usepackage{xcolor}
\usepackage{helvet}
\renewcommand{\familydefault}{\sfdefault}
\usepackage{amsmath,amssymb,amsthm,textcomp}
\usepackage{tikz}
\usepackage[a4paper,head=1.5cm,bottom=2.5cm,left=25mm,right=25mm]{geometry}

\setlength\parindent{0pt}
\newcommand{\titleline}{\rule{\linewidth}{0.5pt}}

\makeatletter
\renewcommand{\maketitle}{
\begin{center}
\vspace{2ex}
{\huge \textsc{\@title}}
\vspace{1ex}
\\
\titleline\\
\@author \hfill \@date
\vspace{3ex}
\end{center}
}
\makeatother
\usepackage{fancyhdr}
\pagestyle{fancy}
\lhead{}
\chead{}
\rhead{}
\lfoot{}
\cfoot{}
\rfoot{Seite \thepage}
\renewcommand{\headrulewidth}{0pt}
\renewcommand{\footrulewidth}{0pt}
\usepackage{listings}

\definecolor{Titelfarbe}{HTML}{CF4A30}
\definecolor{Titelhintergrundfarbe}{RGB}{171, 37, 36}
\definecolor{ivory}{RGB}{255,255,240}
\definecolor{Purple}{HTML}{911146}
\definecolor{Orange}{HTML}{CF4A30}
\definecolor{LightOrange}{HTML}{FFCF9E}
\definecolor{quoteColor}{RGB}{255,255,240}

\lstset{
	keywordstyle=\ttfamily\color{keywordColor}\bfseries, 
	backgroundcolor=\color{ivory},
	commentstyle=\ttfamily\color{red}, 
	stringstyle=\ttfamily,
	numberstyle=\normalsize, 
	basicstyle=\normalsize\ttfamily, 
	showstringspaces=false, 
	breaklines=true, 
	language=Java
}


\title{Übungen: White-Box-Testverfahren}
\author{Markus Flingelli}
\date{\today}

\begin{document}
\maketitle

\section*{Aufgabe 1}
Im Folgenden ist die Methode \lstinline{gcd( int a, int b)}| gegeben, welche den größten gemeinsamen Teiler der ganzen Zahlen $a$ und $b$ berechnet:

\begin{lstlisting}[language=Java]
public static int gcd(int a, int b) {
  int result;
  if (b == 0) {
    result = a;
  } else {
    if (a == 0) {
      result = b;
    } else {
      while (b != 0) {
        if (a > b) {
          a -= b;
        } else {
          b -= a;
        }
      }
      result = a;
    }
  }
  return result;
}
\end{lstlisting}

\subsection*{Aufgabe 1.1}

Implementiere obige Methode in einer Klasse.

\subsection*{Aufgabe 1.2}

Erstelle zu der implementierten Methode in einer eigenen Testklasse alle Testfälle, die notwendig sind, um eine 100\%ige Anweisungsüberdeckung zu erzielen.

\subsection*{Aufgabe 1.3}

Ergänze die Testfälle, die notwendig sind, um eine 100\%ige Entscheidungsüberdeckung zu erzielen.

\subsection*{Aufgabe 1.4}

Worin unterscheiden sich Anweisungs- und Entscheidungsüberdeckung? 

\newpage
\section*{Aufgabe 2}
Die Methode \lstinline{calc}(int value) berechnet die Summe der Zahlen bis $value$:

\begin{lstlisting}[language=Java]
int calcSum(int value) {
  int sum = 0;
  int i = 0;
  while(i < value) {
    sum += ++i;
  }
  return sum;
}
\end{lstlisting}

\subsection*{Aufgabe 2.1}
Wie viele Testfälle sind notwendig, um eine vollständige Pfadabdeckung erzielen zu können?

\subsection*{Aufgabe 2.2}
Was könnten Einschränkungen der vollständigen Pfadabdeckung sein?

\newpage
\section*{Aufgabe 3}
Als Goldener Schnitt wird das Teilungsverhältnis einer Strecke oder anderen Größe bezeichnet, bei dem das Verhältnis des Ganzen zu seinem größeren Teil (auch Major genannt) dem Verhältnis des größeren zum kleineren Teil (dem Minor) gleich ist. Mit $a$ als Major und $b$ als Minor gilt also:

\[  
\frac{a}{b} = \frac{a + b}{a}
\]

Ein Entwickler erstellte eine Methode, die prüft, ob bei ganzzahligen Werten $a$ und $b$ diese den goldenen Schnitt aufweisen: 

\begin{lstlisting}[language=Java]
public static boolean goldenRatio(int a, int b) {
  if (a == 0 || b == 0) {
    return false;
  } else {
    return a*a == b*(a+b);
  }
}
\end{lstlisting}

\subsection*{Aufgabe 3.1}

Implementiere obige Methode in einer Klasse.

\subsection*{Aufgabe 3.2}

Erstelle zu der implementierten Methode in der einer eigenen Testklasse alle Testfälle, die notwendig sind, um eine einfache Bedingungsüberdeckung zu erzielen.

\subsection*{Aufgabe 3.2}

Ergänze jene Testfälle, die notwendig sind, um eine mehrfache Bedingungsüberdeckung zu erzielen.

\end{document}