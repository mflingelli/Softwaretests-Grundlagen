\begin{document}

\only<article>{
  \thispagestyle{empty}
  \pagecolor{white}\afterpage{\nopagecolor}
  \maketitle
}

\begin{frame}[plain]
 \titlepage
\end{frame}

\frame{
\only<presentation>{
  \frametitle{Inhaltsverzeichnis}
}
\tableofcontents[hideallsubsections]
}

\section{Software-Qualit\"at}

\frame{
\frametitle{Software-Qualität}
\begin{block}{Definiton nach ISO 25010}
\begin{quotation}
\glqq Der Grad, zu dem eine Komponente oder ein System die expliziten und impliziten Bedürfnisse seiner verschiedenen Stackholder erfüllt.\grqq
\end{quotation}
\end{block}
}

\frame{
\frametitle{Hauptqualitätsmerkmale von Software}
\begin{center}
\begin{tikzpicture}[scale=.65,transform shape]
  \tikzstyle{level 1 concept}+=[font=\sf, sibling angle=45]
  \path[mindmap,concept color=LightOrange,text=black,every node/.style={concept}]
  node[concept] {Qualität nach ISO 25010} [clockwise from=0]
      child { node[concept] (Functionality) {Funktionali\-tät} }
      child { node[concept] (Reliability) {Zuverlässig\-keit} }
      child { node[concept] (Usability) {Gebrauchs\-tauglich\-keit} }
      child { node[concept] (Security) {IT-Sicherheit} }
      child { node[concept] (Maintainability) {Wartbarkeit} }
      child { node[concept] (Efficiency) {Effizienz} }
      child { node[concept] (Portability) {Portabilität} }
      child { node[concept] (Compatibility) {Kompatibili\-tät} };
\end{tikzpicture}
\end{center}
}

\section{Testen}

\frame{
\frametitle{Testen}
\begin{block}{Definiton}
\begin{itemize}
  \item Prozess, der sämtliche Testaktivitäten umfasst, welche dem Ziel dienen
  \begin{itemize}
    \item Nachweis der vollständigen Umsetzung der Anforderungen
    \item Nachweis, dass System seinen Zweck erfüllt
	\item Erreichen der festgelegten Qualitätsanforderungen
	\item Aufdecken von etwaigen Fehlerzuständen
  \end{itemize}
\end{itemize}
\end{block}
}

\frame{
\frametitle{Testen}
\setbeamercolor{postit}{fg=black,bg=quoteColor}
\begin{beamercolorbox}[sep=1em,wd=.9\textwidth]{postit}
  {\large \grqq Program testing can be a very effective way to show the presence of bugs, but is hopelessly inadequate for showing their absence.\grqq}
  \vskip5mm
  \hspace*\fill{\tiny --- Edsger Wybe Dijkstra, The Humble Programmer, ACM Turing Lecture 1972}
\end{beamercolorbox}
}

\subsection{Testprozess}

\frame{
\frametitle{Fundamentaler Testprozess nach ISTQB}
\begin{center}
\begin{tikzpicture}
\node (Beginn)       at (0,3)  [fill=LightOrange,minimum height=0.75em,text width=1cm,ellipse, text centered] {Beginn};
\node (Planung)      at (0,2)  [rectangle, fill=LightOrange,   text width=6cm, minimum height=1em, text centered, rounded corners=8pt] {Planung und Steuerung};
\node (Analyse)      at (0,1)  [rectangle, fill=LightOrange,   text width=6cm, minimum height=1em, text centered, rounded corners=8pt] {Analyse und Design};
\node (Realisierung) at (0,0)  [rectangle, fill=LightOrange,   text width=6cm, minimum height=1em, text centered, rounded corners=8pt] {Realisierung und Durchf\"uhrung};
\node (Auswertung)   at (0,-1) [rectangle, fill=LightOrange,   text width=6cm, minimum height=1em, text centered, rounded corners=8pt] {Auswertung und Bericht};
\node (Abschluss)    at (0,-2) [rectangle, fill=LightOrange,   text width=6cm, minimum height=1em, text centered, rounded corners=8pt] {Abschluss};
\node (Ende)         at (0,-3) [fill=LightOrange,minimum height=0.75em,text width=1cm,ellipse, text centered] {Ende};
\draw[->, very thick](Beginn) to node[right]{} (Planung);
\draw[->, very thick](Planung) to node[right]{} (Analyse);
\draw[->, very thick](Analyse) to node[right]{} (Realisierung);
\draw[->, very thick](Realisierung) to node[right]{} (Auswertung);
\draw[->, very thick](Auswertung) to node[right]{} (Abschluss);
\draw[->, very thick](Abschluss) to node[right]{} (Ende);
\draw[<-, very thick](Analyse) to[bend left=90] node[right]{} (Auswertung);
\draw[<-, very thick](Realisierung) to[bend left=90] node[right]{} (Auswertung);
\draw[->, very thick](Abschluss) to[bend left=90] node[right]{} (Planung);
\draw[->, very thick](Abschluss) to[bend left=90] node[right]{} (Analyse);
\end{tikzpicture}
\end{center}
}

\subsection{Prinzipien}
\frame{
\frametitle{Allgemeine Prinzipien des Softwaretestens}
\begin{itemize}
	\item Testen zeigt Anwesenheit von Fehlern
	\item Vollständiges Testen ist nicht möglich
	\item Mit dem Testen frühzeitig beginnen
	\item Häufung von Fehlern
	\item Zunehmende Testresistenz
	\item Testen ist abhängig vom Umfeld
	\item Trugschluss: Keine Fehler bedeuten ein brauchbares System
\end{itemize}
}

\subsection{Fehler}

\frame{
\frametitle{Fehler}
\begin{center}
\begin{tikzpicture}
\node (Fehlhandlung) at (0,0)  [rectangle, fill=LightOrange,   text width=6cm, minimum height=1em, text centered, rounded corners=8pt] {Fehlhandlung (error)};
\node (Fehlerzustand) at (0,-1.5)  [rectangle, fill=LightOrange,   text width=6cm, minimum height=1em, text centered, rounded corners=8pt] {Fehlerzustand (bug, defect)};
\node (Fehlerwirkung) at (0,-3)  [rectangle, fill=LightOrange,   text width=6cm, minimum height=1em, text centered, rounded corners=8pt] {Fehlerwirkung (failure)};
\node (Schaden) at (0,-4.5)  [rectangle, fill=LightOrange,   text width=6cm, minimum height=1em, text centered, rounded corners=8pt] {Schaden};
\draw[->, very thick](Fehlhandlung) to node{} (Fehlerzustand);
\draw[->, very thick](Fehlerzustand) to node{} (Fehlerwirkung);
\draw[->, very thick](Fehlerwirkung) to node{} (Schaden);
\end{tikzpicture}
\end{center}
}

\subsection{Debugging}
\frame{
\frametitle{Debugging}
\begin{block}{Definiton}
\begin{itemize}
	\item Prozess der Aufdeckung, Analyse und Entfernung der Ursache von Fehlerwirkungen in einer Komponente oder einem Fehler
	\item Debugging erfolgt in der Regel durch den Entwickler
\end{itemize}
\end{block}
}

\section{Dynamische Testverfahren}
\subsection{Black-Box-Testverfahren}
\frame{
\frametitle{Black-Box-Testverfahren}
\begin{block}{Definiton}
\begin{itemize}
	\item Testverfahren, das auf einer Analyse der Spezifikation einer Komponente oder eines System basiert.
\end{itemize}
\end{block}
}

\frame{
\frametitle{Black-Box-Testverfahren}
\begin{itemize}
	\item Äquivalenzklassenbildung
	\item Grenzwertanalyse
	\item Entscheidungstabellen
	\item Zustandsbasierte Verfahren
	\item Anwendungsfallbasierte Verfahren
\end{itemize}
}

\frame{
\frametitle{Äquivalenzklasse}
\begin{block}{Definiton}
Teil des Wertebereichs von Ein- oder Ausgaben, in dem ein gleichartiges Verhalten der Komponente oder des Systems angenommen wird, basierend auf der zugrunde liegenden Spezifikation
\end{block}
}

\frame{
\frametitle{Grenzwertanalyse}
\begin{block}{Definiton}
\begin{itemize}
  \item Stellt sicher, dass Fehler in kritischen Grenzbereichen der Äquivalenzklassen gefunden werden
\end{itemize}
\end{block}
}

\frame{
\frametitle{Entscheidungstabellen}
\begin{block}{Definiton}
\begin{itemize}
  \item Entscheidungstabellen werden eingesetzt, um komplexe Abhängigkeiten zwischen mehreren Bedingungen und den jeweils auszuführenden Aktionen übersichtlich, vollständig und widerspruchsfrei darzustellen.
\end{itemize}
\end{block}
}

\frame{
\frametitle{Zustandsbasierte Verfahren}
\begin{block}{Definiton}
\begin{itemize}
  \item Testverfahren, bei der Tests aus einer in Form eines Zustandsautomaten vorliegenden Spezifikation abgeleitet werden.
\end{itemize}
\end{block}
}

\frame{
\frametitle{Anwendungsfallbasierte Verfahren}
\begin{block}{Definiton}
\begin{itemize}
  \item Testverfahren, bei dem Testfälle im Hinblick auf die Ausführung verschiedener Verhalten eines Anwendungfalls entworfen werden.
\end{itemize}
\end{block}
}

\subsection{White-Box-Testverfahren}
\frame{
\frametitle{White-Box-Testverfahren}
\begin{block}{Definiton}
\begin{itemize}
	\item Testverfahren, das nur auf der inneren Struktur einer Komponente oder eines Systems basiert.
\end{itemize}
\end{block}
}

\frame{
\frametitle{White-Box-Testverfahren}
\begin{itemize}
	\item Anwesungsüberdeckung
	\item Entscheidungsüberdeckung
	\item Pfadüberdeckung
	\item Einfache und mehrfache Bedingungsüberdeckung
\end{itemize}
}

\frame{
\frametitle{Anwesungsüberdeckung}
\begin{block}{Definiton}
\begin{itemize}
  \item Jede Anweisung wird mindestens einmal getestet
  \item Deckt nicht erreichbare Anweisungen (dead code) im Quelltext auf
  \item Bei Verzweigungen (z.B. Schleifen, Bedigungen) werden Datenabhängigkeiten nicht beachtet
\end{itemize}
\end{block}
}

\frame{
\frametitle{Entscheidungsüberdeckung}
\begin{block}{Definiton}
\begin{itemize}
  \item Durchläuft alle Zweige des Quellcodes
  \item 100 \% Entscheidungsüberdeckung schließt 100 \% der Anweisungen ein
\end{itemize}
\end{block}
}

\frame{
\frametitle{Pfadüberdeckung}
\begin{block}{Definiton}
\begin{itemize}
  \item Beim Pfadüberdeckungstest werden im Kontrollflussgraphen die möglichen Pfade vom Startknoten bis zum Endknoten betrachtet. 
\end{itemize}
\end{block}
}

\frame{
\frametitle{Einfache und mehrfache Bedingungsüberdeckung}
\begin{block}{Definiton}
\begin{itemize}
  \item \textbf{Einfache Bedingungsüberdeckung}:\newline Jede atomare Bedingung einer Entscheidung muss einmal mit true und einmal mit false getestet werden.
  \item \textbf{Mehrfache Bedingungsüberdeckung}:\newline Dieser Test betrachtet alle atomaren Bedingungen einer Bedingung. Wenn $n$ atomare Bedingungen in der Bedingung stehen, dann werden $2^n$ Kombinationen gebildet. 
\end{itemize}
\end{block}
}

\subsection{Vergleich Black- und White-Box-Testverfahren}
\frame{
\frametitle{Vergleich Black- und White-Box-Testverfahre}
\begin{itemize}
  \item Vorteile von Black-Box-Tests gegenüber White-Box-Tests 
  \begin{itemize}
    \item bessere Verifikation des Gesamtsystems
	\item Testen von bedeutungsmäßigen Eigenschaften bei geeigneter Spezifikation
	\item Portabilität von systematisch erstellten Testsequenzen auf plattformunabhängige Implementierungen
  \end{itemize}
  \item Nachteile von Black-Box-Tests gegenüber White-Box-Tests
  \begin{itemize}
    \item größerer organisatorischer Aufwand
	\item zusätzlich eingefügte Funktionen bei der Implementierung werden nur durch Zufall getestet
	\item Testsequenzen einer unzureichenden Spezifikation sind unbrauchbar
  \end{itemize}
\end{itemize}
}

\section{Teststufen}

\frame{
\frametitle{Teststufen}
\begin{block}{Teststufen}
\begin{itemize}
	\item Komponententest
	\item Integrationstest
	\item Systemtest
	\item Abnahmetest
\end{itemize}
\end{block}
}

\frame{
\frametitle{Komponententest}
\begin{block}{Definiton}
\begin{itemize}
  \item Testet einzele Software- oder Hardware-Komponente mit dem Ziel 
  \begin{itemize}
    \item Fehler aufzudecken und
	\item vollständige Umsetzung der funktionalen Anforderungen an die Komponente nachzuweisen
  \end{itemize}
  \item Komponententests von Software-Komponenten werden in der Regel vom Entwickler erstellt
\end{itemize}
\end{block}
}

\frame{
\frametitle{Integrationstest}
\begin{block}{Definiton}
\begin{itemize}
  \item Prüft das korrekte funktionale und technische Zusammenspiel zwischen Komponten oder Systemen
  \item Soll Fehlerwirkungen in den Schnittstellen und Kommunikation zwischen den Komponenten aufdecken
\end{itemize}
\end{block}
}

\frame{
\frametitle{Systemtest}
\begin{block}{Definiton}
\begin{itemize}
  \item Teststufe mit dem Schwerpunkt zu verifizieren, dass ein System als Ganzes die spezifizierten Anforderungen erfüllt
\end{itemize}
\end{block}
}

\frame{
\frametitle{Systemtest}
\begin{block}{Definiton}
\begin{itemize}
  \item Teststufe mit dem Schwerpunkt zu verifizieren, dass ein System als Ganzes die spezifizierten Anforderungen erfüllt
  \item Wird in der Regel von Testern ausgeführt
\end{itemize}
\end{block}
}

\frame{
\frametitle{Abnahmetest}
\begin{block}{Definiton}
\begin{itemize}
  \item Teststufe mit dem Schwerpunkt zu bestimmen, ob ein System abgenommen werden kann
  \item Abnahmetest erfolgt mit dem Auftraggeber
\end{itemize}
\end{block}
}

\section{Literaturverzeichnis}


\frame{
\frametitle{Literaturverzeichnis}
\begin{thebibliography}{10}
  \bibitem{GTB} German Testing Board: \emph{Standard Glossary of Terms used in Software Testing}. Version 3.3,  \href{https://www.german-testing-board.info/wp-content/uploads/2020/05/CTFL_CORE_SAMPLE_EXAM_2018A_germ_mL_mB_FINAL.pdf}{\url{https://www.german-testing-board.info/wp-content/uploads/2020/05/CTFL_CORE_SAMPLE_EXAM_2018A_germ_mL_mB_FINAL.pdf}}, 25. Februar 2020
\end{thebibliography} 
}

\end{document}